\begin{figure*}[b]
\centering
\begin{subfigure}[b]{0.5\columnwidth}
\centering
\resizebox {\columnwidth} {!} {
\begin{tikzpicture}
\begin{axis}[
width=7cm, height=5cm,
xlabel=Number of epochs,
ylabel=Training loss,
yticklabel style={
	/pgf/number format/fixed,
	/pgf/number format/precision=5
},
%ymax=0.001,
ymin=0,
xmin=-1,
axis lines=left
]
\addplot [color=black, line width=1pt] table [x index=0, y index=1] {Data/new_epsilon2}; \label{epsilonP_SW}
\addplot [color=red, line width=1pt] table [x index=0, y index=2] {Data/new_epsilon2}; \label{epsilonP_Q2}
\addplot [color=green, line width=1pt] table [x index=0, y index=3] {Data/new_epsilon2}; \label{epsilonP_Q4}
\addplot [color=blue, line width=1pt] table [x index=0, y index=4] {Data/new_epsilon2}; \label{epsilonP_Q8}
\node [
fill=white,
font=\footnotesize,
anchor=north west
] at (rel axis cs: 0.01,0.99) {\shortstack[l]{
\ref{epsilonP_SW} \textit{float}\\
%CPU runtime: 2.86s, FPGA runtime: 0.677s\\
\ref{epsilonP_Q2} \textit{Q2}\\
%FPGA runtime: 0.0967s\\
\ref{epsilonP_Q4} \textit{Q4}\\
%FPGA runtime: 0.186s\\
\ref{epsilonP_Q8} \textit{Q8} %\\
%FPGA runtime: 0.36s
}};
\end{axis}
\end{tikzpicture}
}
\caption{Effect of different precision quantization on \textit{epsilon}. $\gamma=1/2^{12}$.}
\label{figure:epsilonP}
\end{subfigure}
\begin{subfigure}[b]{0.5\columnwidth}
\centering
\resizebox {\columnwidth} {!} {
\begin{tikzpicture}
\begin{axis}[
width=7cm, height=5cm,
xlabel=Number of epochs,
ylabel=Training loss,
yticklabel style={
	/pgf/number format/fixed,
	/pgf/number format/precision=5
},
ymode=log,
ymax=0.14,
ymin=0.001,
xmin=-1,
axis lines=left
]
\addplot [color=black, line width=1pt] table [x index=0, y index=1] {Data/new_synthetic1003}; \label{syn100S_CPU9}
\addplot [color=red, line width=1pt] table [x index=0, y index=2] {Data/new_synthetic1003}; \label{syn100S_Q29}
\addplot [color=black, dashed, line width=1pt] table [x index=0, y index=3] {Data/new_synthetic1003}; \label{syn100S_CPU12}
\addplot [color=red, dashed, line width=1pt] table [x index=0, y index=4] {Data/new_synthetic1003}; \label{syn100S_Q212}
\addplot [color=black, dotted, line width=1pt] table [x index=0, y index=5] {Data/new_synthetic1003}; \label{syn100S_CPU15}
\addplot [color=red, dotted, line width=1pt] table [x index=0, y index=6] {Data/new_synthetic1003}; \label{syn100S_Q215}

\node [
fill=white,
font=\footnotesize,
anchor=north east
] at (rel axis cs: 1.08,0.99) {\shortstack[l]{
\ref{syn100S_CPU9} \textit{float}, $\gamma=1/2^9$, \ref{syn100S_Q29} \textit{Q2}, $\gamma=1/2^9$\\
\ref{syn100S_CPU12} \textit{float}, $\gamma=1/2^{12}$, \ref{syn100S_Q212} \textit{Q2}, $\gamma=1/2^{12}$\\
\ref{syn100S_CPU12} \textit{float}, $\gamma=1/2^{15}$, \ref{syn100S_Q212} \textit{Q2}, $\gamma=1/2^{15}$
}};
\end{axis}
\end{tikzpicture}
}
\caption{Effect of decreasing the step size on \textit{Q2} with \textit{synthetic100}.}
\label{figure:syn100S}
\end{subfigure}
\begin{subfigure}[b]{0.5\columnwidth}
\centering
\resizebox {\columnwidth} {!} {
\begin{tikzpicture}
\begin{axis}[
width=7cm, height=5cm,
xlabel=Number of epochs,
ylabel=Training loss,
yticklabel style={
	/pgf/number format/fixed,
	/pgf/number format/precision=5
},
%ymode=log,
%ymax=1000,
ymin=0,
%xmin=-1,
axis lines=left
]
\addplot [color=black, line width=1pt] table [x index=0, y index=1] {Data/new_synthetic1004}; \label{synthetic100I_SW}
\addplot [color=red, line width=1pt] table [x index=0, y index=2] {Data/new_synthetic1004}; \label{synthetic100I_Q2_64}
\addplot [color=blue, line width=1pt] table [x index=0, y index=3] {Data/new_synthetic1004}; \label{synthetic100I_Q2_32}
\addplot [color=green, line width=1pt] table [x index=0, y index=4] {Data/new_synthetic1004}; \label{synthetic100I_Q2_16}
\addplot [color=cyan, line width=1pt] table [x index=0, y index=5] {Data/new_synthetic1004}; \label{synthetic100I_Q2_8}
\addplot [color=red, dashed, line width=1pt] table [x index=0, y index=6] {Data/new_synthetic1004}; \label{synthetic100I_Q2_4}
\addplot [color=blue, dashed, line width=1pt] table [x index=0, y index=7] {Data/new_synthetic1004}; \label{synthetic100I_Q2_2}
\node [
fill=white,
font=\footnotesize,
anchor=north east
] at (rel axis cs: 1.02,0.99) {\shortstack[l]{
\ref{synthetic100I_SW} \textit{float} \\
\ref{synthetic100I_Q2_64} \textit{Q2}, 64 indexes, 
\ref{synthetic100I_Q2_32} \textit{Q2}, 32 indexes\\
\ref{synthetic100I_Q2_16} \textit{Q2}, 16 indexes,
\ref{synthetic100I_Q2_8} \textit{Q2}, 8 indexes\\
\ref{synthetic100I_Q2_4} \textit{Q2}, 4 indexes,
\ref{synthetic100I_Q2_2} \textit{Q2}, 2 indexes
}};
\end{axis}
\end{tikzpicture}
}
\caption{Effect of reusing indexes with \textit{synthetic100}.}
\label{figure:synthetic100I}
\end{subfigure}
\begin{subfigure}[b]{0.5\columnwidth}
\centering
\resizebox {\columnwidth} {!} {
\begin{tikzpicture}
\begin{axis}[
width=7cm, height=5cm,
xlabel=Number of epochs,
ylabel=Training loss,
yticklabel style={
	/pgf/number format/fixed,
	/pgf/number format/precision=5
},
ymax=0.3,
ymin=0.04,
xmin=-1,
axis lines=left
]
\addplot [color=black, line width=1pt] table [x index=0, y index=1] {Data/new_epsilon_effect_of_Q}; \label{epsilon5_SW}
\addplot [color=red, line width=1pt] table [x index=0, y index=2] {Data/new_epsilon_effect_of_Q}; \label{epsilon5_Q2}
\addplot [color=green, line width=1pt] table [x index=0, y index=3] {Data/new_epsilon_effect_of_Q}; \label{epsilon5_Q4}
\addplot [color=blue, line width=1pt] table [x index=0, y index=4] {Data/new_epsilon_effect_of_Q}; \label{epsilon5_Q8}
\addplot [color=red!70!black, dashed, line width=1pt] table [x index=0, y index=5] {Data/new_epsilon_effect_of_Q}; \label{epsilon5_F2}
\addplot [color=green!70!black, dashed, line width=1pt] table [x index=0, y index=6] {Data/new_epsilon_effect_of_Q}; \label{epsilon5_F4}
\addplot [color=blue!70!black, dashed, line width=1pt] table [x index=0, y index=7] {Data/new_epsilon_effect_of_Q}; \label{epsilon5_F8}
\node [
fill=white,
font=\footnotesize,
anchor=north west
] at (rel axis cs: 0.01,0.99) {\shortstack[l]{
\ref{epsilon5_SW} \textit{float}\\
\ref{epsilon5_Q2} \textit{Q2}, \ref{epsilon5_F2} \textit{F2}\\
\ref{epsilon5_Q4} \textit{Q4}, \ref{epsilon5_F4} \textit{F4}\\
\ref{epsilon5_Q8} \textit{Q8}, \ref{epsilon5_F8} \textit{F8} %\\
}};
\end{axis}
\end{tikzpicture}
}
\caption{Naive rounding vs. stochastic quantization}
\label{figure:naive}
\end{subfigure}
\caption{SGD on various data sets, showing the effects of data precision, step size $\gamma$, index reuse, and naive rounding.}
\label{figure:effects}
\vspace{-1em}
\end{figure*}
\begin{figure*}[h!]
\centering
\begin{subfigure}[t]{.66\columnwidth}
\centering
\resizebox {\columnwidth} {!} {
\begin{tikzpicture}
\begin{axis}[
width=7cm, height=5cm,
xlabel=Number of epochs,
ylabel=Training loss,
yticklabel style={
	/pgf/number format/fixed,
	/pgf/number format/precision=5
},
%ymode=log,
%ymax=1000,
ymin=0,
%xmin=-1,
axis lines=left
]
\addplot [color=black, line width=1pt] table [x index=0, y index=1] {Data/new_synthetic1004}; \label{synthetic1004_SW}
\addplot [color=red, line width=1pt] table [x index=0, y index=2] {Data/new_synthetic1004}; \label{synthetic1004_Q2_64}
\addplot [color=blue, line width=1pt] table [x index=0, y index=3] {Data/new_synthetic1004}; \label{synthetic1004_Q2_32}
\addplot [color=green, line width=1pt] table [x index=0, y index=4] {Data/new_synthetic1004}; \label{synthetic1004_Q2_16}
\addplot [color=cyan, line width=1pt] table [x index=0, y index=5] {Data/new_synthetic1004}; \label{synthetic1004_Q2_8}
\addplot [color=red, dashed, line width=1pt] table [x index=0, y index=6] {Data/new_synthetic1004}; \label{synthetic1004_Q2_4}
\addplot [color=blue, dashed, line width=1pt] table [x index=0, y index=7] {Data/new_synthetic1004}; \label{synthetic1004_Q2_2}
\node [
fill=white,
font=\tiny,
anchor=north east
] at (rel axis cs: 0.99,0.99) {\shortstack[l]{
\ref{synthetic1004_SW} \textit{float} \\
\ref{synthetic1004_Q2_64} \textit{Q2}, 64 indexes\\
\ref{synthetic1004_Q2_32} \textit{Q2}, 32 indexes\\
\ref{synthetic1004_Q2_16} \textit{Q2}, 16 indexes\\
\ref{synthetic1004_Q2_8} \textit{Q2}, 8 indexes\\
\ref{synthetic1004_Q2_4} \textit{Q2}, 4 indexes\\
\ref{synthetic1004_Q2_2} \textit{Q2}, 2 indexes
}};
\end{axis}
\end{tikzpicture}
}
\caption{SGD on \textit{synthetic100}.}
\label{figure:synthetic1004}
\end{subfigure}
\begin{subfigure}[t]{.66\columnwidth}
\centering
\resizebox {\columnwidth} {!} {
\begin{tikzpicture}
\begin{axis}[
width=7cm, height=5cm,
xlabel=Number of epochs,
ylabel=Training loss,
yticklabel style={
	/pgf/number format/fixed,
	/pgf/number format/precision=5
},
ymode=log,
%ymax=1000,
ymin=0.005,
%xmin=-1,
axis lines=left
]
\addplot [color=black, line width=1pt] table [x index=0, y index=1] {Data/new_gisette4}; \label{gisette4_SW}
\addplot [color=red, line width=1pt] table [x index=0, y index=2] {Data/new_gisette4}; \label{gisette4_Q2_64}
\addplot [color=blue, line width=1pt] table [x index=0, y index=3] {Data/new_gisette4}; \label{gisette4_Q2_32}
\addplot [color=green, line width=1pt] table [x index=0, y index=4] {Data/new_gisette4}; \label{gisette4_Q2_16}
\addplot [color=cyan, line width=1pt] table [x index=0, y index=5] {Data/new_gisette4}; \label{gisette4_Q2_8}
\addplot [color=red, dashed, line width=1pt] table [x index=0, y index=6] {Data/new_gisette4}; \label{gisette4_Q2_4}
\addplot [color=blue, dashed, line width=1pt] table [x index=0, y index=7] {Data/new_gisette4}; \label{gisette4_Q2_2}
\node [
fill=white,
font=\tiny,
anchor=north east
] at (rel axis cs: 0.99,0.99) {\shortstack[l]{
\ref{gisette4_SW} \textit{float} \\
\ref{gisette4_Q2_64} \textit{Q2}, 64 indexes\\
\ref{gisette4_Q2_32} \textit{Q2}, 32 indexes\\
\ref{gisette4_Q2_16} \textit{Q2}, 16 indexes\\
\ref{gisette4_Q2_8} \textit{Q2}, 8 indexes\\
\ref{gisette4_Q2_4} \textit{Q2}, 4 indexes\\
\ref{gisette4_Q2_2} \textit{Q2}, 2 indexes
}};
\end{axis}
\end{tikzpicture}
}
\caption{SGD on \textit{gisette}.}
\label{figure:gisette4}
\end{subfigure}
\begin{subfigure}[t]{.66\columnwidth}
\centering
\resizebox {\columnwidth} {!} {
\begin{tikzpicture}
\begin{axis}[
width=7cm, height=5cm,
xlabel=Number of epochs,
ylabel=Training loss,
yticklabel style={
	/pgf/number format/fixed,
	/pgf/number format/precision=5
},
%ymode=log,
ymax=0.8,
ymin=0.03,
%xmin=-1,
axis lines=left
]
\addplot [color=black, line width=1pt] table [x index=0, y index=1] {Data/new_epsilon4}; \label{epsilon4_SW}
\addplot [color=red, line width=1pt] table [x index=0, y index=2] {Data/new_epsilon4}; \label{epsilon4_Q2_64}
\addplot [color=blue, line width=1pt] table [x index=0, y index=3] {Data/new_epsilon4}; \label{epsilon4_Q2_32}
\addplot [color=green, line width=1pt] table [x index=0, y index=4] {Data/new_epsilon4}; \label{epsilon4_Q2_16}
\addplot [color=cyan, line width=1pt] table [x index=0, y index=5] {Data/new_epsilon4}; \label{epsilon4_Q2_8}
\addplot [color=red, line width=1pt] table [x index=0, y index=6] {Data/new_epsilon4}; \label{epsilon4_Q2_4}
\addplot [color=blue, line width=1pt] table [x index=0, y index=7] {Data/new_epsilon4}; \label{epsilon4_Q2_2}
\node [
fill=white,
font=\tiny,
anchor=north west
] at (rel axis cs: 0.01,0.99) {\shortstack[l]{
\ref{epsilon4_SW} \textit{float} \\
\ref{epsilon4_Q2_64} \textit{Q2}, 64 indexes\\
\ref{epsilon4_Q2_32} \textit{Q2}, 32 indexes\\
\ref{epsilon4_Q2_16} \textit{Q2}, 16 indexes\\
\ref{epsilon4_Q2_8} \textit{Q2}, 8 indexes\\
\ref{epsilon4_Q2_4} \textit{Q2}, 4 indexes\\
\ref{epsilon4_Q2_2} \textit{Q2}, 2 indexes
}};
\end{axis}
\end{tikzpicture}
}
\caption{SGD on \textit{epsilon}.}
\label{figure:epsilon4}
\end{subfigure}
\caption{Effect of using less indexes than the number of performed epochs. More than 16 indexes is usually enough to get the same convergence quality.}
\label{figure:indexes}
\end{figure*}